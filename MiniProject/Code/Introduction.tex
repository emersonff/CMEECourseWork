
\section{Introduction}

With the rapidly increasing of computing power over the recent years, model selection has become a more preferred method for data analysis and biological inferences rather than the other relatively traditional method, the null hypothesis testing\citep{JOHNSON2004101}. It based on the likelihood theory and is capable of making inferences from a set of competing hypotheses. Compared to the null hypothesis testing, model selection is not restricted in evaluating a single model where significance is measured against some arbitrary probability threshold. Instead, it quantitatively measures models by ranking and weighting each model. Thus, it has been commonly used in biological related industries. Specifically, as a promising research field among food microbiologists, it has been adopted widely in the food industry for the estimation of bacteria growth patterns \citep{Baranyi1993,Garthright1991}. It allows estimation of the shelf life of foods and describes how different environmental conditions such as temperature and carbon sources affect the behaviour of pathogenic bacteria\citep{MCCLURE1994265}. According to Whiting, predictive models in microbiology can be divided into three levels\citep{whiting1995microbial}. Level one predictive models describe the number of cells as a function of time. Secondary level models summarize the effect of environmental conditions. And the tertiary models combine the first two levels.

Typically, bacterial growth in batch culture can be considered as consisting of four stages when the growth curve is defined as the natural logarithm of cell concentration against time: the lag phase, exponential growth phase, stationary phase and mortality phase \citep{mckellar2004primary}. During the lag phase, the cells adapt themselves to the new environments and express alternative metabolic capabilities. The cells need sufficient time for metabolic machinery to generate enough energy that is needed for cell replication\citep{Buchanan1997}. After the cells have adapted, they grow at a maximum growth rate, which is often denoted as $\mu$, under the specific environment. Once the carbon sources become limited due to depletion or other reasons, the stationary phase is reached. The population has now reached the carrying capacity of the specific environment. Accordingly, the specific growth rate($\mu$) during the stationary phase is equal to 0. Finally, the mortality phase is reached. the growth curve indicates a decline in population size as microbes die because of the lackness of essential nutrition and increasing of waste products. These four phases compose a sigmoidal curve.\\

The objective of this study is to fit previously collected microbial growth data to 4 published level one models mentioned in the following section using NLLS method and then statistically determine which model(s) best fits the growth data mainly using model selection approaches. \\

